\documentclass[12pt]{article}

% text definitions
\newcommand{\acronym}[1]{\small{{#1}}}
\newcommand{\project}[1]{\textsl{{#1}}}
\newcommand{\gaia}{\project{Gaia}}
\newcommand{\tgas}{\project{\acronym{TGAS}}}

% math definitions

% style
\linespread{1.09}
\sloppy
\sloppypar
\raggedbottom
\frenchspacing

\begin{document}

\section*{Finding old, disrupted stellar systems in \\ \textsl{Gaia TGAS}}

\noindent
APW \& DWH

\paragraph{Abstract:}
% context
The first data release from the \gaia\ Mission includes the
\tgas\ Catalog, which---despite being low-precision and small relative
to \gaia's end-of-mission data products---is the most precise, large
catalog of proper motions and parallaxes ever made.
This size and precision permit the discovery of kinematic structures
in the Milky Way that could not have been found previously.
% aims
We aim to find disrupting stellar systems by identifying pairs of
stars that have high probability of being on orbits that intersected
at a space-time event (a $3+1$-dimensional origin event) at some time in the past.
The analysis is complicated by the noisy parallax and proper-motion
measurements, and the heterogeneous radial-velocity data; even stars
that \emph{do} have a common origin event might not appear to, given
the noise.
% method
Our method involves integrating phase-space points backwards in time
in a model of the Milky Way gravitational potential, accounting for
non-trivial posterior beliefs about phase-space by integrating many
posterior samples per star.
Pairs with common origin events will appear in our analysis as
having---at some point in the past---substantial overlap of their sets
of integrated phase-space points.
We assess the false-positive rate and set thresholds by, among other things, integrating
forwards, finding stars that will intersect in the future.
% results
From a parent sample of YYY stars in \tgas\ with radial-velocity
measurements, and making conservative cuts, we find XXX pairs with
common origin events.
These stars have ZZZ properties and appear to be YYY.

\section{Introduction}

Whatevs.

\end{document}
